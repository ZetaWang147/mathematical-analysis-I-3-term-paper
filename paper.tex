\documentclass{article}
\usepackage{fontspec}  % 字体选择
\usepackage{xeCJK}     % 中文支持
\usepackage{hyperref}  % 超链接支持
\usepackage{cite}      % 参考文献=
\usepackage{amsthm}%提供proof环境
\usepackage{amsmath}
\usepackage{amsfonts}
\usepackage{amssymb}
\usepackage{graphicx}
\usepackage{siunitx}
\usepackage{enumitem}%修改列表格式

\title{度量空间}
\author{ZetaWang}
\date{\today}

\begin{document}

\maketitle

\begin{abstract}
本文旨在通过介绍度量空间及其拓扑性质,为将来拓扑学的学习进行铺垫。
\end{abstract}

\tableofcontents % 生成目录

\section{度量空间的介绍}
\textbf{定义1.1.}给定一个集合 $X$,把他的元素称为点(point),如果存在一个映射 $d:X\times X\rightarrow \mathbb{R}$,对任意 $x,y,z\in X$,满足:
\begin{enumerate}[label={\textbullet}]
    \item $d(x,y)\geq 0$,且 $d(x,y) = 0$ 当且仅当 $x=y$;
    \item $d(x,y) = d(y,x)$;
    \item $d(x,y) + d(y,z) \geq d(x,z)$。
\end{enumerate}
则称 $X$ 为度量空间(metric space),$d$ 为距离(distance) .\setlength{\parskip}{1em} % 在段落之间增加额外的垂直间距

以下为一些度量空间的例子\setlength{\parskip}{1em} % 在段落之间增加额外的垂直间距

\textbf{例1.2.}验证欧氏空间及其度量构成一个度量空间。
\begin{proof}
    \leavevmode
    \begin{enumerate}[label={\textbullet}]
        \item $d(x,y) = (\sum\limits_{k=1}^{n}( x_k-y_k)^2)^{\frac{1}{2}} \geq 0$,且 $d(x,y) = 0$ 当且仅当 $x=y$;
        \item $d(x,y) = d(y,x)$;
        \item 由 Minkowski 不等式,
        \begin{align*}
            % d(x,y) + d(y,z)
            & = (\sum_{k=1}^{n}( x_k-y_k)^2)^\frac{1}{2} + (\sum_{k=1}^{n}( y_k-z_k)^2)^{\frac{1}{2}} \\
            & \geq (\sum_{k=1}^{n}( x_k-z_k)^2)^{\frac{1}{2}} \\
            & = d(x,z)
        \end{align*}
    \end{enumerate}
    因此,欧氏空间及其度量构成一个度量空间。
\end{proof}

\textbf{例1.3.}设 $E$ 是一个非空集合,对任意的 $x,y\in E$,定义
$$d(x,y) = \begin{cases} 0 & x=y \\ 1 & x\neq y \end{cases}$$
证明 $d$ 是 $E$ 上的度量。拥有这种度量的集合 $E$ 称为离散度量空间(discrete metric space).

\textbf{例1.4.}设 $E$ 是以 $d$ 为度量的度量空间,对任意 $x,y\in E$,定义
$$d_1(x,y) = \frac{d(x,y)}{1+d(x,y)}$$
证明 $d_1$ 也是 $E$ 上的一个度量。

\textbf{例1.5.}设 $E$ 是定义在有界闭区间 $[a,b]$ 上的全体有界函数所成之集,对任意 $f,g\in E$,定义
$$d(f,g) = \sup_{x\in [a,b]}|f(x)-g(x)|$$.
证明 $d$ 是 $E$ 上的度量。

\textbf{例1.6.}设 $p$ 是一个素数,
对任一非零整数 $a$,
定义 $v_p(a)$ 为使得 $p^k$ 整除 $a$ 的最大的 $k$,
即 $v_p(a) = \max\{k\in\mathbb{N}: p^k\mid a\}$.
进一步,对非零有理数 $x = \frac{a}{b}$($a$ 和 $b$ 均为整数),
定义 $v_p(x) = v_p(a) - v_p(b)$.现对 $x\in\mathbb{Q}$,定义
$$\vert x\vert_p = \begin{cases} p^{-v_p(x)} & x\neq 0 \\ 0 & x=0 \end{cases}$$
证明 $d(x,y) = \vert x-y\vert_p$ 是 $\mathbb{Q}$ 上的度量。

\textbf{定义1.7.}集合 $E$ 上的度量 $d$ 是 non-Archimedean ,是指对任意 $x,y,z\in E$,有
$$d(x,z) \leq max\{d(x,y),d(y,z)\}$$
否则,称 $d$ 为 Archimedean .

容易验证,例1.2、例1.5中的度量都 Archimedean ,
而例1.6中的度量 non-Archimedean .

\textbf{例1.8.}设 $E$ 是一个具有 non-Archimedean 度量的度量空间。对任意 $a\in E$ 及 $r>0$,
称 $B(a,r) = \{x\in E: d(x,a)<r\}$ 为以 $a$ 为圆心,$r$ 为半径的圆。证明:圆内每个点均为该圆的圆心。

\section{度量空间的拓扑结构}
\textbf{定义2.1.}设 $X$ 为度量空间。
\begin{enumerate}[label={\textbullet}]
    \item 对 $p\in X$,称 $N_r(p) = \{q\in X: d(p,q)\leq r\}$ 为 $p$ 的邻域(neighbourhood),$r$ 为半径(radius);
    \item 
\end{enumerate}
\section*{参考文献} % 无编号的参考文献部分
\bibliographystyle{plain} % 参考文献样式
\bibliography{references}  % 指向参考文献文件(references.bib)

\end{document}
