\documentclass{article}
\usepackage{fontspec}  % 字体选择
\usepackage{xeCJK}     % 中文支持
\usepackage{hyperref}  % 超链接支持
\usepackage{cite}      % 参考文献=

\usepackage{amsthm}%提供proof环境
\usepackage{amsmath}
\usepackage{amsfonts}
\usepackage{amssymb}
\usepackage{graphicx}
\usepackage{siunitx}

\title{度量空间}
\author{ZetaWang}
\date{\today}

\begin{document}

\maketitle

\begin{abstract}
本文旨在通过介绍度量空间及其拓扑性质,为将来拓扑学的学习进行铺垫。
\end{abstract}

\tableofcontents % 生成目录

\section{度量空间的介绍}
给定一个集合 $X$,把他的元素称为点(point),如果存在一个映射 $d:X\times X\rightarrow \mathbb{R}$,对任意 $x,y,z\in X$,满足:
\begin{enumerate}
    \item $d(x,y)\geq 0$,且 $d(x,y) = 0$ 当且仅当 $x=y$;
    \item $d(x,y) = d(y,x)$;
    \item $d(x,y) + d(y,z) \geq d(x,z)$。
\end{enumerate}
则称 $X$ 为度量空间(metric space),$d$ 为距离(distance) .\setlength{\parskip}{1em} % 在段落之间增加额外的垂直间距

以下为一些度量空间的例子:\setlength{\parskip}{1em} % 在段落之间增加额外的垂直间距

\textbf{例1}验证欧氏空间及其度量构成一个度量空间。(即 $d(x,y) = (\sum_{k=1}^{n}( x_k-y_k)^2)^{\frac{1}{2}}$)
\begin{proof}
    \begin{enumerate}
        \item $d(x,y) = (\sum_{k=1}^{n}( x_k-y_k)^2)^{\frac{1}{2}} \geq 0$,且 $d(x,y) = 0$ 当且仅当 $x=y$;
        \item $d(x,y) = d(y,x) $;
        \item 由 Minkowski 不等式,
        \begin{align*}
            % d(x,y) + d(y,z)
            & = (\sum_{k=1}^{n}( x_k-y_k)^2)^\frac{1}{2} + (\sum_{k=1}^{n}( y_k-z_k)^2)^{\frac{1}{2}} \\
            & \geq (\sum_{k=1}^{n}( x_k-z_k)^2)^{\frac{1}{2}} \\
            & = d(x,z)
        \end{align*}
    \end{enumerate}
    因此,欧氏空间及其度量构成一个度量空间。
\end{proof}

\textbf{例2}设 $E$ 是一个非空集合,对任意的 $x,y\in E$,定义
$$d(x,y) = \begin{cases} 0 & x=y \\ 1 & x\neq y \end{cases}$$
证明 $d$ 是 $E$ 上的度量。拥有这种度量的集合 $E$ 称为离散度量空间(discrete metric space).

\section{度量空间的拓扑结构}


\section*{参考文献} % 无编号的参考文献部分
\bibliographystyle{plain} % 参考文献样式
\bibliography{references}  % 指向参考文献文件(references.bib)

\end{document}
